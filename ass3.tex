\documentclass{article}
\usepackage[bottom=1in,top=1.5in]{geometry}

\usepackage{paralist}
\usepackage{enumerate}
\usepackage{xspace}
\usepackage{amsmath}
\usepackage{amsfonts}
\usepackage{algorithm}
\usepackage{algpseudocode}
\usepackage{hyperref}
\usepackage{comment}

\usepackage{listings}
\usepackage{xcolor}


\definecolor{keywords}{rgb}{0.7,0.1,0.5}
\definecolor{strings}{rgb}{0.1,0.7,0.1}
\definecolor{comments}{rgb}{0.1,0.1,0.7}


\lstset{
  language=Python,
  basicstyle=\footnotesize\ttfamily,
  keywordstyle=\color{keywords},
  stringstyle=\color{strings},
  commentstyle=\color{comments},
}

\newcommand{\ordo}{\mathcal{O}}
\newcommand{\mset}[1]{\lbrace #1 \rbrace}


\title{Assignment 3 \\ Algorithms and Data Structures 1 (1DL210) \\ 2020}
\author{Hooman Asadian, Sarbojit Das}

\begin{document}

\maketitle
\section{Instructions}
\begin{itemize}
\item Unless you have asked for and received special permission, solutions must be prepared in groups.  If you got special permission for any previous assignment, you also have special permission for this assignment.
\item If you are not using the skeleton code, make sure that your program behaves similarly.  If you are using a language that does not allow functionsas arguments, you may write one copy of \verb|insert|, \verb|search| and \verb|delete| for each hash function, or you can solve it in some other way.
\item When you have completed the assignment, you should have three files: one program with all required functions (including the tests) for hash tables,one program with all required function for BSTs, and one PDF documentexplaining the results in Section 2.3. Include your names in the programfiles.
\item Only one person in each group should submit the solution.
\end{itemize}

\section{Hash Tables}
For this assignment, you are going to implement and experiment with hash tables.
You will write functions for insertion, deletion and searching using chaining.

\subsection{Skeleton Code}

Again, we provide some skeleton code in Python. You are of course free to implement the algorithms in the language of your choice as long as
your solution meets the requirements. The file {\tt ass3ht.py} (for Python 2.7.x) and {\tt ass3ht\_python3.py} (for Python 3.x.x) contains the following functions:

\begin{description}

\item \lstinline{ht(n)}
Takes an integer $n$ and returns an empty hash table \footnote{For simplicity, the given function {\tt ht} yields a hash table which does not have the same time complexity  as ``real'' hash tables for insertion, seaching, etc. This is easy to fix by switching the
underlying data structures.} 
with $n$ buckets. 

\item \lstinline{output(t)} 
Prints the contents of the hash table \lstinline{t}. If any bucket contains more than 30 elements, the number of elements is printed instead.

\item \lstinline{randstring(n)}
Produces a random DNA string of length $2n$. 

\item \lstinline{h1(e)} 
The hash function $37e \mod 11$.

\item \lstinline{h2(e)} 
The hash function $30e \mod 8$.

\item \lstinline{h3(e)}
The hash function $e^2 \mod 23$.

\item \lstinline{h4(e)}
The hash function $\sum_{i=1}^n |e_i| \mod 19$, where $e_i$ is the $i$:th element of \lstinline{e} and $|e_i|$ denotes the ASCII value of $e_i$.
In other words, the sum of the character values of all characters in the string, modulo 19.

\end{description}

\subsection{Implementation {\bf (3p)}}

Assuming collision resolution by {\bf chaining}, implement the following functions:

\begin{enumerate}
\item \lstinline{insert(t, h, e)}

Inserts the element \lstinline{e} into the hash table \lstinline{t} using hash function \lstinline{h}. Collisions are resolved by chaining (\lstinline{list.apppend(e)} in Python).

\item \lstinline{search(t, h, e)}

Returns \lstinline{True} if the element \lstinline{e} exists in the hash table \lstinline{t} and \lstinline{False} otherwise. Assumes \lstinline{h} was used for hashing.

\item \lstinline{delete(t, h, e)}

Deletes the element \lstinline{e} from the hash table \lstinline{t}. Assumes \lstinline{h} was used for hashing.

\end{enumerate}

If you prefer, you may instead create a data structure for the hash table and store the hash function there. In this case, you don't need to pass
the hash function to the functions above.

\subsection{Tests {\bf (4p + 1p)}}

Now that you have implemented a hash table, it is time to try out some hash functions.
In this part of the assignment, you will write some tests and explain the results. The explanations should be brief.

\begin{enumerate}
\item Write a function \lstinline{test1} that creates an empty hash table with 11 buckets, inserts 1000 random\footnote{You can use \lstinline{random.randint(0, n)} to get a random integer $x$ s.t. $0 \leq x \leq n$}
 elements using \lstinline{insert} and \lstinline{h1} and then ouputs the
table using \lstinline{output}. How does the hash function perform? Explain the results.

\item Write a function \lstinline{test2} that creates an empty hash table with 8 buckets, inserts 1000 random
 integers (between 0 and 1000) using \lstinline{insert} and \lstinline{h2} and then ouputs the
table using \lstinline{output}. How does the hash function perform? Explain the results.

\item (Bonus) Write a function \lstinline{test3} that creates an empty hash table with 23 buckets, inserts 1000 random
 integers using \lstinline{insert} and \lstinline{h3} and then ouputs the
table using \lstinline{output}. How does the hash function perform? Explain the results. 

\item Write a function \lstinline{test4} that creates an empty hash table with 11 buckets, 
inserts the elements $\mset{1, 12, 23, 34, 45, 56, 67, 78, 89, 100, 111}$
using \lstinline{insert} and \lstinline{h1} and then ouputs the
table using \lstinline{output}. How does the hash function perform? Explain the results.

\item Write a function \lstinline{testdna} that creates an empty hash table with 19 buckets, 
inserts 1000 random DNA strings (generated by e.g. \lstinline{randstring(10)}) using \lstinline{insert} and \lstinline{h4}
and then outputs the result. How does the hash function perform? Explain the results.

\end{enumerate}

\section{Binary Search Trees}

In the final part of this assignments, you will write some functions on binary search trees.
Binary search trees are data structures used for storing key-value pairs (i.e. dictionaries).
A binary search tree consists of either nothing (i.e. the empty tree, denoted by $\bot$), or a triple $(x,l,r)$,
where $x$ represents the key contained in the root, and $l,r$ represent the left and right subtrees, respectively.
It must satisfy the following properties:
\begin{itemize}
\item All keys are distinct
\item For all keys $x'$ in $l$, $x' < x$
\item For all keys $x'$ in $r$, $x' > x$
\item $l$ and $r$ are BSTs
\end{itemize}
For simplicity, we will only store the keys, and not care about the data.
In this manner, a tree which contains the element $1$ is represented by the triple $(1, \bot, \bot)$,
and the tree which is constructed by inserting $2,6,7,4,1$ (in this particular order) into an empty tree
is represented by the triple $(2, (1, \bot, \bot), (6, (4, \bot, \bot), (7, \bot, \bot)))$\footnote{Please draw this tree to make sure this is correct.}. 

In Python, we can use \lstinline{None} to represent the empty tree. The above trees are then represented by 
\begin{lstlisting}
(1, None, None)
\end{lstlisting}
and
\begin{lstlisting}
(2, (1, None, None), (6, (4, None, None), (7, None, None))).
\end{lstlisting}

We define the functions \lstinline{emptytree}, which returns an empty tree, and \lstinline{insert(t, e)}, 
which returns the tree obtained by inserting \lstinline{e} into \lstinline{t} in the following way:

\begin{lstlisting}
def emptytree():
    return None
\end{lstlisting}

\begin{lstlisting}
def insert(t, x):
    if t == None:
        return (x, None , None)
    else:
        key, left, right = t
        if x == key:
            return (key, left, right)
        elif x < key:
            return (key, insert(left, x), right)
        else:
            return (key, left, insert(right, x))
\end{lstlisting}

We can now create arbitrary binary search trees by repeatedly calling \lstinline{insert}. Run the following to print the previous examples:

\begin{lstlisting}
from ass3bst import *

t1 =  insert(emptytree(), 1)
print t1  # python 2.7.x
# print(t1) # python 3.x.x

t2 = insert(emptytree(), 2)
t2 = insert(t2, 6)
t2 = insert(t2, 7)
t2 = insert(t2, 4)
t2 = insert(t2, 1)
print t2 # for python 2.7.x
# print(t2) # for python 3.x.x

\end{lstlisting}

For convenience, we include the function \lstinline{treefromlist(l)}, which takes a list and inserts the elements, in order, into an empty tree.

\begin{lstlisting}
from ass3bst import *

t3 = treefromlist([10,3,5,1])
print t3 # for python 2.7.x
# print(t3) # for python 3.x.x

> (10, (3, (1, None, None), (5, None, None)), None)
\end{lstlisting}

The above code can be found in {\tt ass3bst.py} (for both 2.7.x and 3.x.x).

\subsection{Implementation (3p + 1p)}

This part of the assignment consists of two parts.
Note that if you implement your solution in Python, you {\bf must} use the representation of BSTs given above. 

\begin{enumerate}

\item
{\bf (1p)}
(Bonus) Implement the function \lstinline{inorderwalk(t)} which, given a tree \lstinline{t}, 
prints the elements of \lstinline{t} in ascending order. Example:

\begin{lstlisting}
from ass3bst import *

t3 = treefromlist([10,3,5,1])
inorderwalk(t3)

> 1 3 5 10
\end{lstlisting}

\item
{\bf (3p)}
Implement the function \lstinline{delete(t, x)}, which returns the tree  \lstinline{t} with \lstinline{x} removed. Example: 

\begin{lstlisting}
from ass3bst import *

t4 = treefromlist([2,6,7,4,1])
t4 = delete(t4, 6)
print t4 # for python 2.7.x
# print(t4) # for python 3.x.x

> (2, (1, None, None), (7, (4, None, None), None))

t5 = insert(emptytree(), 1)
t5 = delete(t5, 1)
print t5 # for python 2.7.x
# print(t5) # for python 3.x.x
> None
\end{lstlisting}
%
{\bf Hint:} Removing leaves is trivial. What happens when you remove a node for which its subtrees are non-empty?
\end{enumerate}
\end{document}
